\documentclass[11pt, twocolumn, final]{article}
\usepackage[sc]{mathpazo}
\usepackage[T1]{fontenc}
\linespread{1.05}
\usepackage{lettrine}
%\documentclass[paper=a4, fontsize=11pt,twoside]{scrartcl}
\usepackage{cite}
\usepackage{blindtext}
\usepackage{booktabs}
\usepackage[UKenglish]{babel}
\usepackage[level]{datetime}
\usepackage{lastpage}
\usepackage{graphicx}
\usepackage{makecell}
\usepackage{tabularx}
\usepackage{mwe}
%\graphicspath{{C:/Bestanden/Dropbox/Kenneth eigen bestanden/School/Researchpractica/Differentiaal vergelijkingen RP/Images/}}
\usepackage[rflt]{floatflt}
\usepackage[a4paper,text={15.8cm,25.2cm},centering]{geometry}
%\usepackage{pgfornament,tkzexample,tikzrput}     
\newdateformat{datekennethstyle}{%
\centerfloat
{\footnotesize   \hspace{1mm}\ordinal{DAY} of \monthname[\THEMONTH], \THEYEAR}
}%
\usepackage{fancyhdr}
\usepackage{amssymb}

\fancyhf{}
\pagestyle{fancy}

%Pagestyle voor abstract etc
\makeatletter
\newcommand\ps@AbstractPageStyle{%
  \fancyhf{}
  %\fancyhead[L]{\hspace{10mm} \includegraphics[scale=0.2, trim=0mm 0mm 0mm 0mm, clip]{PICerasmuslogo2.PNG}}
% \fancyhead[L]{\hspace{8.90mm} \includegraphics[scale=0.909, trim=0mm -0.1mm 0mm 0mm, clip]{Erasmuslogoklein.PNG}}
  %\fancyhead[R]{Dani\"el den Hoed Cancer Institute\hspace{9.59mm}}
  \fancyfoot[R]{\thepage}
  \renewcommand{\headrulewidth}{1.5pt}
  \renewcommand{\footrulewidth}{0.4pt}
}

%Pagestyle voor eerste pagina van sectie
\makeatletter
\newcommand\ps@SectiePagina{%
  \fancyhf{}
%\fancyhead[R]{Erasmus Medical Centre\hspace{9.59mm}}
\fancyfoot[R]{Page \thepage\ of \pageref{LastPage}}
 \renewcommand{\headrulewidth}{0pt}
  \renewcommand{\footrulewidth}{0.4pt}
}

%Pagestyle voor Titlepage
\makeatletter
\newcommand\ps@TitlePageStyle{%
  \fancyhf{}

\makeatletter
    \def\headrule{{\if@fancyplain\let\headrulewidth\plainheadrulewidth\fi
        \hrule\@height\headrulewidth\@width 0in \vskip-\headrulewidth}}
\makeatother
%\fancyhead[LO]{\rightmark}
%\fancyhead[RO]{\makebox[0pt][l]{\makebox[6cm][r]{\thepage}}}
%\fancyhead[L]{\hspace*{8.80mm} \includegraphics[scale=0.1485, trim=0mm -1mm 0mm 0mm, clip]{ErasmusMCBW2.png}}
%\fancyhead[R]{\hspace*{-12mm} \includegraphics[scale=0.282, trim=0mm -0.5mm -35.3mm 0mm, clip]{HHSLogo.PNG}}

\makeatletter
    \def\footrule{{\if@fancyplain\let\footrulewidth\plainfootrulewidth\fi
        \hrule\@height\footrulewidth\@width 0in \vskip-\footrulewidth}}
\makeatother

% \renewcommand{\headrulewidth}{0pt}
% \fancyhead[L]{\hspace{8.90mm} \includegraphics[scale=0.5, trim=0mm 20mm 0mm 10mm, clip]{Erasmuslogo.PNG}}
% \fancyhead[R]{\hspace{8.90mm} \includegraphics[scale=0.17, trim=0mm 0mm 0mm 0mm, clip]{HHSLogo.PNG}}


}


%Pagestyle voor normale paginas
\makeatletter
\newcommand\ps@Normaal{%
  \fancyhf{}
  \fancyhead[L]{\hspace{10mm}\nouppercase{\leftmark}}
  %\fancyhead[R]{Erasmus Medical Centre\hspace{9.59mm}}
  \fancyfoot[R]{Page \thepage\ of \pageref{LastPage}}
  \renewcommand{\headrulewidth}{1.5pt}
  \renewcommand{\footrulewidth}{0.4pt}
}

%Pagestyle voor Aftermatter
\makeatletter
\newcommand\ps@Aftermatter{%
  \fancyhf{}
  %\fancyhead[L]{\hspace{10mm} \includegraphics[scale=0.2, trim=0mm 0mm 0mm 0mm, clip]{PICerasmuslogo2.PNG}}
 %\fancyhead[L]{\hspace{8.90mm} \includegraphics[scale=0.17, trim=4.9mm 26mm 9mm 30mm, clip]{PICerasmuslogo.PNG}}
  %\fancyhead[R]{Dani\"el den Hoed Cancer Institute\hspace{9.59mm}}
  \fancyfoot[R]{Page \thepage\ of \pageref{LastPage}}
  \renewcommand{\headrulewidth}{1.5pt}
  \renewcommand{\footrulewidth}{0.4pt}
}

% Length to control the \fancyheadoffset and the calculation of \headline
% simultaneously
 \renewcommand{\headrulewidth}{1.05pt}

\newlength\FHoffset
\setlength\FHoffset{1cm}

\addtolength\headwidth{2\FHoffset}

\fancyheadoffset{\FHoffset}

% these lengths will control the headrule trimming to the left and right 
\newlength\FHleft
\newlength\FHright

% here the trimmings are controlled by the user
\setlength\FHleft{1.01cm}
\setlength\FHright{0.768cm}

% The new definition of headrule that will take into acount the trimming(s)
\newbox\FHline
\setbox\FHline=\hbox{\hsize=\paperwidth%
  \hspace*{\FHleft}%
  \rule{\dimexpr\headwidth-\FHleft-\FHright\relax}{\headrulewidth}\hspace*{\FHright}%
}
\renewcommand\headrule{\vskip-.7\baselineskip\copy\FHline}


\pagestyle{Normaal}

%Distance between references
\newcommand{\RefDist}{2.19mm}


\usepackage{algorithmicx}

%\usepackage{pgfplots}
%\usepackage{titlesec} % for chapter formatting

%\usepackage[space]{grffile}
%\usepackage{graphicx}
%\usepackage{makecell}

%PAD VOOR IMAGES!!
%\graphicspath{{F:/Home/KGH/Verslag/Verslag/Figures/}}
\usepackage[rflt]{floatflt}
\usepackage{anyfontsize}
%\usepackage[a4paper,pdftex]{geometry}

\usepackage{a4wide}
\usepackage[margin=10pt,font=footnotesize,labelfont=bf, labelsep=endash]{caption} % labelfont=bf,
\usepackage{eurosym}
\usepackage{todonotes}

\usepackage{colortbl}
\newcommand{\gray}{\rowcolor[gray]{.96}}

\usepackage{lscape}
\usepackage[siunitx]{circuitikz}
\usepackage{url} % suggested by Wikipedia for more nicely formatted web addresses.
\usepackage[pdfstartview=FitV,colorlinks=true,linkcolor=black,citecolor=black,urlcolor=black,linktocpage=all]{hyperref}
\usepackage{float}
%\usepackage{pdflatex}
\usepackage{epstopdf}
\usepackage{color}
\usepackage{transparent}
\usepackage{afterpage}
\usepackage{multirow}
%\usepackage{pstricks}
%\usepackage{pst-node}
%\usepackage{pst-plot}
%\usepackage{pst-circ}
%\usepackage[table]{xcolor}
\usepackage{array}
\usepackage{caption}
\usepackage{fixltx2e}
%\usepackage{subcaption}
\usepackage{pdfpages}
\usepackage[bottom]{footmisc}
\usepackage{etoolbox}
\usepackage{tocloft}
\usepackage{subfig}

\renewcommand{\cftsecfont}{\LARGE\bfseries}

\makeatletter
\renewcommand*\l@section{\@dottedtocline{1}{0em}{1.5em}}
\makeatother

\setcounter{tocdepth}{2}

%\usepackage[nottoc,numbib]{tocbibind}

\makeatletter
\newcommand*{\centerfloat}{%
  \parindent \z@
  \leftskip \z@ \@plus 1fil \@minus \textwidth
  \rightskip\leftskip
  \parfillskip \z@skip}
\makeatother

\usepackage[protrusion=true,expansion=true]{microtype}	
\usepackage{amsmath,amsfonts,amsthm,amssymb}

%Aanpassen definitie array voor matrix stelsels!!!!
%Aanpassen definitie array voor matrix stelsels!!!!
\makeatletter
\renewcommand*\env@matrix[1][*\c@MaxMatrixCols c]{%
  \hskip -\arraycolsep
  \let\@ifnextchar\new@ifnextchar
  \array{#1}}
\makeatother


\usepackage{graphicx}
\pdfoptionpdfminorversion 6

\makeatletter
\renewcommand\section{\@startsection{section}{1}{\z@}%
                                  {-3.5ex \@plus -1ex \@minus -.2ex}%
                                  {2.3ex \@plus.2ex}%
                                  {\normalfont\LARGE\bfseries}}
\makeatother

\makeatletter
\renewcommand\subsection{\@startsection{subsection}{1}{0.0mm}%
                                  {-1.5ex \@plus -1ex \@minus -.2ex}%
                                  {1.3ex \@plus.2ex}%
			{\large\bfseries}}
\makeatother


\makeatletter
\renewcommand\paragraph{\@startsection{paragraph}{4}{\z@}%
  {-3.25ex\@plus -1ex \@minus -.2ex}%
  {1.5ex \@plus .2ex}%
  {\normalfont\normalsize\bfseries}}
\makeatother

%\newcommand{\sectionlineFancy}{
%  \nointerlineskip \vspace{3mm}
%  \hspace{\fill}    \rput[c](-3pt,3pt){\pgfornament[scale=.17]{75}}   \hspace{\fill}
%  \par\nointerlineskip
%}

\newcommand{\sectionlineBeginEen}{
  \nointerlineskip \vspace{3mm}
  \hspace{\fill}\rule{0.192\linewidth}{.01pt}\hspace{\fill}
  \par\nointerlineskip
}

\newcommand{\sectionlineBeginTwee}{
  \nointerlineskip 
  \hspace{\fill}\rule{0.115\linewidth}{.01pt}\hspace{\fill}
  \par\nointerlineskip
}

\newcommand{\sectionlineEind}{
  \nointerlineskip
  \hspace{\fill}\rule{0.72\linewidth}{.7pt}\hspace{\fill}
  \par\nointerlineskip \vspace{\baselineskip}
}

%\renewcommand{\familydefault}{\sfdefault}
\frenchspacing

\widowpenalty=10000
\clubpenalty=10000

\renewcommand{\textfraction}{0.00}
\renewcommand{\topfraction}{0.99}
\renewcommand{\bottomfraction}{0.99}
\renewcommand{\floatpagefraction}{0.35}

\setcounter{totalnumber}{5}
\topmargin=-15.mm		% beyond 25.mm
\oddsidemargin=0.mm		% beyond 25.mm
\evensidemargin=0.mm		% beyond 25.mm
\headheight=35pt
\headsep=7mm
\textheight=230.mm
\textwidth=160.mm
\parskip=3.08mm
\parindent=0mm
%\pagestyle{fancyplain}
%\cfoot{\fontfamily{sf}\thepage}
\usepackage{mathtools}
\mathtoolsset{showonlyrefs}


\renewcommand{\baselinestretch}{1.2}
\newcommand{\mat}[1]{\mathbf{#1}} % vector and matrix notation

%\bibliographystyle{unsrt}
\usepackage[square,numbers,super]{natbib}
 

%\makeatletter 
%\renewcommand\NAT@citesuper[3]{\5ifNAT@swa 
%\unskip\kern\p@\textsuperscript{\NAT@@open #1\NAT@@close}% 
%\if*#3*\else\ (#3)\fi\else #1\fi\endgroup} 
%\makeatother 

\renewcommand{\thefootnote}{\fnsymbol{footnote}}

%% Define a new 'leo' style for the package that will use a smaller font.
\makeatletter
\def\url@leostyle{%
  \@ifundefined{selectfont}{\def\UrlFont{\sf}}{\def\UrlFont{\small\ttfamily}}}
\makeatother
%% Now actually use the newly defined style.
\urlstyle{leo}