\section{Introduction}
The Fermi energy of a material is the difference between the highest and lowest electron-energies of the material at $T=0~\text{K}$. The goal of this research project is to measure the Fermi energy of aluminium, using a commonly applied method called the Angular Correlation of Annihilation Radiation (ACAR)\cite{Nascimento2009244}. ACAR relies on the fact that the angle between the two photons that are created during electron-positron annihilation is a measure for the momentum of the electron prior to the annihilation process. Since the Fermi energy of a material dictates the maximum momentum of the electron, the minimum angle found between the two photons is a measure for the Fermi energy.
For this research project, a $^{22}$Na source was used which emits positrons that annihilated primarily with the free electrons in the aluminum surrounding the source. Two scintillation detectors were positioned on opposite sides of the source. The angle between the two detectors was varied and for each angle the number of simultaneous photons was counted. A theoretical curve was fitted to the measured data, from which the Fermi energy was calculated.