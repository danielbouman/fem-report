\section{Experimental Setup}
In this experiment a $^{22}$Na is used as a positron source (details about $^{22}$Na are discussed in Appendix \ref{ap:na22}) and is positioned in between two scintillation detectors, both equipped with a photomultiplier tube (PMT) and an amplifier. Due to the source emitting primarily positrons, positron-electron annihilation will occur in the surrounding aluminium. The scintillation detectors allow detection of the $\gamma$-radiation resulting from electron-positron annihilations. A motorized stage is used to displace one of the scintillation detectors over a small angle, allowing measurements across an angular range. Computer software is used to control the stage and acquire the data corresponding to the angles. Using an array of electronics only simultaneous 511~keV photons are registered. Over a period of five days the data is collected. The data is then fit to a fitting function with several parameters.
\begin{figure}[H]
\centering
\resizebox{\columnwidth}{!}{
% Drawing Views
\tikzstyle{mainview}=[x={(0.610cm,-0.410cm)},y={(0cm,0.820cm)},z={(-0.990cm,-0.410cm)}]

  \begin{tikzpicture}[
  	mainview,
  	arrow/.style={black,->,>=stealth',shorten >=1pt},
    transition/.style={black,->,>=stealth',shorten >=1pt},
    photon/.style={transition,decorate,decoration={snake,amplitude=1.5}}
  ]
  
	% Coordinates
	\coordinate (O) at (0,0,0);

	% Angles
    
    \draw[arrow,dashed,shorten >=0pt] (O) ..controls (.2,0,-.02) and (.8,0,.16) .. (1.3,0,.53);
    \draw[arrow,dashed] (O) ..controls (-.2,0,-.02) and (-.8,0,.16) .. (-1.3,0,.53);
    
			
	%% Detector 1
    % Connector
    \draw (O) .. controls (-.3,0,-.2) and (.2,0,-.5) .. (.7,0,-.8);
	\draw[fill=gray!40] (0,0,0) circle (0.3); 
  	\fill[fill=gray!40] (-0.21,0.22,0) 
    	.. controls (-0.23,0.22,0.5) .. (-0.25,0.44,1) -- (0.25,-0.44,1)  
        .. controls (0.23,-0.22,0.5) .. (0.21,-0.22,0) -- cycle;
	\draw (-0.21,0.22,0) .. controls (-0.23,0.22,0.5) .. (-0.25,0.44,1); 
	\draw (0.21,-0.22,0) .. controls (0.23,-0.22,0.5) .. (0.25,-0.44,1); 
	\draw[fill=gray!20] (0,0,1) circle (0.5);
   	\draw[arrow] (-2,0,.05) node[above] {Detector 1} -- (-.6,0,.1); %Label
	
    %% Collimator 1
    \draw[fill=gray,opacity=0.7] (-.05,-.5,1)--(-.05,-.5,1.3)
    --(-.05,.5,1.3)--(-.05,.5,1)-- cycle;
    \draw[fill=gray,opacity=0.7] (-.05,-.5,1.3)--(-.05,.5,1.3)
    --(-.5,.5,1.3)--(-.5,-.5,1.3)-- cycle;
    \draw[fill=gray,opacity=0.7] (-.05,.5,1.3)--(-.05,.5,1)
    --(-.5,.5,1)--(-.5,.5,1.3)-- cycle;
    
    \draw[fill=gray,opacity=0.7] (.5,-.5,1)--(.5,-.5,1.3)
    --(.5,.5,1.3)--(.5,.5,1)-- cycle;
    \draw[fill=gray,opacity=0.7] (.05,-.5,1.3)--(.05,.5,1.3)
    --(.5,.5,1.3)--(.5,-.5,1.3)-- cycle;
    \draw[fill=gray,opacity=0.7] (.05,.5,1.3)--(.05,.5,1)
    --(.5,.5,1)--(.5,.5,1.3)-- cycle;
    
   	%% Gamma radiation
    \draw[photon] (0,0,2.3) node[below] {$\gamma$, 511~keV} -- (0,0,1.3);
    \draw[photon] (0,0,6.7) node[right] {$\gamma$, 511~keV} -- (0,0,7.7);
     
    
    %% Collimator 2
    \draw[fill=gray,opacity=0.7] (-.05,-.5,7.7)--(-.05,-.5,8)
    --(-.05,.5,8)--(-.05,.5,7.7)-- cycle;
    \draw[fill=gray,opacity=0.7] (-.05,-.5,8)--(-.05,.5,8)
    --(-.5,.5,8)--(-.5,-.5,8)-- cycle;
    \draw[fill=gray,opacity=0.7] (-.05,.5,8)--(-.05,.5,7.7)
    --(-.5,.5,7.7)--(-.5,.5,8)-- cycle;
    
    \draw[fill=gray,opacity=0.7] (.5,-.5,7.7)--(.5,-.5,8)
    --(.5,.5,8)--(.5,.5,7.7)-- cycle;
    \draw[fill=gray,opacity=0.7] (.05,-.5,8)--(.05,.5,8)
    --(.5,.5,8)--(.5,-.5,8)-- cycle;
    \draw[fill=gray,opacity=0.7] (.05,.5,8)--(.05,.5,7.7)
    --(.5,.5,7.7)--(.5,.5,8)-- cycle;
    
	%% Detector 2
	\draw[fill=gray!40,opacity=0.9] (0,0,8) circle (0.5);
  	\fill[fill=gray!40,opacity=0.9] (-0.21,0.22,9) 
    	.. controls (-0.23,0.22,8.5) .. (-0.31,0.39,8) -- (0.25,-0.44,8)  
        .. controls (0.23,-0.22,8.5) .. (0.21,-0.22,9) -- cycle;
	\draw (-0.21,0.22,9) .. controls (-0.23,0.22,8.5) .. (-0.31,0.39,8); 
	\draw (0.21,-0.22,9) .. controls (0.23,-0.22,8.5) .. (0.25,-0.44,8); 
	\draw[fill=gray!20,opacity=0.90] (0,0,9) circle (0.3);
    % Connector
	\draw[fill=white] (0,0,9) circle (0.05);
    \draw (0,0,9) .. controls (-.3,0,9.2) and (.2,0,9.5) .. (.3,0,10);
   	\draw[arrow] (-2,0,8.1) node[above] {Detector 2} -- (-.62,0,8.43); %Label
	
	% Source 
	\draw[fill=gray!50] (-.5,0,4.5) .. 
    	controls (-.5,0,4.7775) and (-0.2775,0,5) .. (0,0,5) .. 
    	controls (0.2775,0,5) and (.5,0,4.7775) .. (.5,0,4.5) .. 
    	controls (.5,0,4.2225) and (0.2775,0,4) .. (0,0,4) .. 
    	controls (-0.2775,0,4) and (-.5,0,4.2225) .. (-.5,0,4.5) -- cycle;
    \fill[fill=gray!50] (-.2,0,4.96) -- (.2,0,4.04) 
    	-- (.2,1.5,4.04) -- (-.2,1.5,4.96) -- cycle;
	\draw (-.2,0,4.96) -- (-.2,1.5,4.96);
	\draw (.2,0,4.04) -- (.2,1.5,4.04);
	\draw[fill=gray!30] (-.5,1.5,4.5) .. 
    	controls (-.5,1.5,4.7775) and (-0.2775,1.5,5) .. (0,1.5,5) .. 
    	controls (0.2775,1.5,5) and (.5,1.5,4.7775) .. (.5,1.5,4.5) .. 
    	controls (.5,1.5,4.2225) and (0.2775,1.5,4) .. (0,1.5,4) .. 
    	controls (-0.2775,1.5,4) and (-.5,1.5,4.2225) .. (-.5,1.5,4.5) -- cycle;
    % Slit   
	\draw[fill=gray!20] (.08,0,4.99) -- (.08,.6,4.99) -- (-.08,.6,4.99) -- (-.08,0,4.99);
    \draw (-.08,0,4.99) -- (-.08,0,4.89);
   	\draw[arrow] (-2,1.5,4.3) node[above] {$^{22}$Na source in lead shielding} -- (-.62,1.5,4.5); %Label
    
\end{tikzpicture} %
}
\caption{Experimental setup. The source and Al sample are enclosed by lead shielding with exit slits towards the detectors. Collimators are used to shield the detectors from photons with incident angles other than perpendicular to the detector surface. Detector 1 can be rotated along the dashed lines.}
\label{fig:setup}
\end{figure}

\subsection*{$^{22}$Na source and scintillation detectors}
The $^{22}$Na source is deposited on an aluminium substrate through physical vapor deposition and on top of this another piece of aluminum is placed, enclosing the $^{22}$Na source. The aluminium substrate is then encapsulated in a solid plastic cylinder, to prevent direct exposure of the source.

High-energy photons from the positron-electron annihilation can propagate through two slits on opposite sides of the lead shielding around the source (Figure \ref{fig:setup}). Photons which are not incident perpendicular with respect to the detector surface are absorbed by lead collimators in front of the scintillator detectors. Since the collimators have a non-zero width, the photons that are measured still have a small range of incident angles. Since one slit acts as rectangular spatial filter, two slits give a triangular function through convolution of the two rectangular functions. Convolution with the fitting function will better approximate the measured data.

The remaining photons that are incident on the scintillator will excite electrons from the scintillator material and lose energy for each excitation. The excited electrons will fall back to a lower energy state, emitting low-energy photons in the process. The amount of photons is proportional to the energy of the initial incident photon. These photons are incident on a photo-cathode inside the PMT which is coupled to the scintillation crystal. Through the photo-electric effect, the low-energy photons will free electrons on from the cathode. The number of freed electrons is proportional to the amount of absorbed low-energy photons, and thus proportional to the energy of the initial incident high-energy photon. Further, the photo-cathode is a thin sheet, which would allow a high-energy photon to pass through the cathode and possibly not free any electrons at all.

The PMT also contains an array of dynodes, which acts as an electron multiplier. The dynodes have a large positive voltage (in the order of kV's) and each consecutive dynode has a larger voltage (around 100~V). The freed electrons are directed and focused toward the first dynode. Accelerated by the electric field due to the potential difference, the electrons arrive each with a kinetic energy of around 100~eV. Upon striking the dynode more electrons are freed. The electrons are then accelerated towards the second dynode where the same process starts again. The arrangement of the dynodes is such that a cascade occurs where at each dynode the number of electrons increases. At the end of the tube the electrons reach the anode, where the electrons produce a sharp current pulse proportional to the energy of the incident photon.

For the angle dependent measurements, detector 1 can be rotated around a range of angles (Figure \ref{fig:setup}). The detector is rotated with a stepping motor, controlled by the measurement software.

\subsection*{Signal processing}
The pulse from the PMT attached to detector~1 is acquired by a single channel analyser (SCA). When the SCA recognizes a current from the PMT, it sends a logical pulse to a time-to-amplitude converter (TAC), which starts a 200~ns timer upon receiving this pulse. If a pulse from detector~2 is received within a certain time frame, a signal is sent to an analog-to-digital converter (ADC). The ADC sends a signal to the computer which registers this as an annihilation event corresponding to the current angle of detector~1.

Two photons from an annihilation process will arrive at the detectors at roughly the same time. Thus to ensure that the TAC has enough time to register the pulse from detector~2 after the timer starts, a time delay between detector~2 and its SCA is created by using a significantly longer coax cable. The length is chosen such that the electromagnetic waves propagating through the dielectric insulator of the coaxial cable arrive within the 200~ns time frame. This ensures that photons from the same annihilation are measured and, because at most 1 annihilation takes place every microsecond (source activity at the moment of experiment is approximately 1~MBq), photons from different annihilation events are excluded.

The detectors are not only sensitive to 511~keV photons. The 1274,5~keV $\gamma$-radiation from the $^{22}$Na source as well as background radiation is also detected (see Figure \ref{fig:bg22na} in appendix \ref{ap:bg22na} for the unfiltered measured spectrum). To exclude these photons from the measurements, acquisition windows on the SCA's are calibrated to only allow 511~keV photons. Figure \ref{fig:bg_511} in appendix \ref{ap:bg_511} shows a detector histogram with only the isolated 511~keV photons.

When the ADC send a signal to the computer, the software registers this event. After an interval the software writes the number of measured annihilations in a spreadsheet with the corresponding angle. At the same time the step motor moves detector~1 to the next angle, repeating this process for the full length of the experiment.

\subsection*{Experiment details}
Because the majority of the annihilation photons are not detected, the length of the experiment has to be relatively long. To obtain enough photons for a proper statistical analysis, the experiment ran 120.6~ hours. During this time the angle between detector~1 and the normal $\theta$ is changed by 3.3~$^{\circ}$. At every angle data is acquired for 2160~s. The setup was calibrated using a less active positron source.