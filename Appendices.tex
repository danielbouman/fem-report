\section{Simulation source code}
\subsection{Mesh}\label{ap:mesh}
\begin{lstlisting}
# square.msh
#
# example file for the generation of elements in a square
# See the SEPRAN Introduction Section 4.2
#
# To run this example use:
#
# sepmesh square.msh
#
constants
  reals
    b = 1			# width of the outer square
    b_half = b/2		# half width of the outer square
    b_0 = 0.4			# width of the inner square
    h = 1			# height of the outer square
    h_0 = 0.4			# height of the inner square
    x_inner_left = b/2-b_0/2
    y_inner_below = h/2-h_0/2
  integers
    n = 50			# number of elements Omega_A horizontal direction
    n_half = n/2		# half number of elements Omega_A horizontal direction
    m = n*(h/b)			# number of elements Omega_A vertical direction
    m_half = m/2		# half number of elements Omega_A vertical direction
    n_0 = n*b_0			# number of elements Omega_B horizontal direction
    m_0 = m*h_0			# number of elements Omega_B vertical direction
end
#
# Definition of the mesh
#
mesh2d
  # Definition of the coordinates
  points
    p1=(0,0)
    p2=(0,h)
    p3=(b_half,h)
    p4=(b_half,y_inner_below+h_0)
    p5=(x_inner_left,y_inner_below+h_0)
    p6=(x_inner_left,y_inner_below)
    p7=(b_half,y_inner_below)
    p8=(b_half,0)
  
  # Definition of the curves  
  curves
    # Boundary curves
    c1 = line ( p1,p2,nelm = m )
    c2 = line ( p2,p3,nelm = n_half )
    c3 = line ( p3,p4,nelm = (m-m_0)/2 )
    c4 = line ( p4,p5,nelm = n_0/2 ) 
    c5 = line ( p5,p6,nelm = m_0 )
    c6 = line ( p6,p7,nelm = n_0/2 )
    c7 = line ( p7,p8,nelm = (m-m_0)/2 )
    c8 = line ( p8,p1,nelm = n_half )
    c9 = line ( p7,p4,nelm = m_0 )
    
  # Definition of the surface  
  surfaces
    s1 = triangle 3 (c1,c2,c3,c4,c5,c6,c7,c8) # Omega_A
    s2 = triangle 3 (c4,c5,c6,c9) # Omega_B

  # Couple each surface to a different element group in order to provide
  # different properties to the coefficients
  meshsurf
      selm1 = s1
      selm2 = s2

  # Plot the mesh  
  plot
end
\end{lstlisting}

\subsection{Elements subroutine}\label{ap:elem}
\begin{lstlisting}
subroutine elemsubr ( npelm, x, y, nunk_pel, elem_mat,  &
                            elem_vec, elem_mass, prevsolution, itype )
!
!                       INPUT / OUTPUT PARAMETERS
!
      implicit none
      integer, intent(in) :: npelm, nunk_pel, itype
      double precision, intent(in) :: x(1:npelm), y(1:npelm),  &
                                      prevsolution(1:nunk_pel)
      double precision, intent(out) :: elem_mat(1:nunk_pel,1:nunk_pel),  &
                                       elem_vec(1:nunk_pel),  &
                                       elem_mass(1:nunk_pel)

! **********************************************************************
!
!                       LOCAL PARAMETERS
!
! ======================================================================
!     
      double precision :: beta(1:3), gamma(1:3), delta, T_inf, suv(3,3), k, alpha
      integer :: i, j
      
! ======================================================================
!
      T_inf = 0d0			!--- Environment temperature
      alpha = 10d0			!--- Heat transfer coefficient
      
      if ( itype==1 ) then    		!--- Internal element for Omega_A

!--- Heat conductivity
	k = 0.1d0
      
!	  Compute the factors beta, gamma and delta as defined in the NMSC
	delta = (x(2)-x(1))*(y(3)-y(1))-(y(2)-y(1))*(x(3)-x(1))
	beta(1) = (y(2)-y(3))/delta
	beta(2) = (y(3)-y(1))/delta
	beta(3) = (y(1)-y(2))/delta
	gamma(1) = (x(3)-x(2))/delta
	gamma(2) = (x(1)-x(3))/delta
	gamma(3) = (x(2)-x(1))/delta
	
!     --- Fill the element matrix as defined in the Lecture Notes

	do j = 1, 3
	  do i = 1, 3
	    elem_mat(i,j) = 0.5d0*k*abs(delta)*(beta(i)*beta(j)+gamma(i)*gamma(j))
	  end do
	end do
	
!     --- The internal element vector is zero
  
	elem_vec(1:3) = 0d0
	
      else if ( itype==2 ) then	 	!--- Internal element for Omega_B
      
!--- Heat conductivity
	k = 1d0
      
!--- Type = 1: internal element
!	  Compute the factors beta, gamma and delta as defined in the NMSC
	delta = (x(2)-x(1))*(y(3)-y(1))-(y(2)-y(1))*(x(3)-x(1))
	beta(1) = (y(2)-y(3))/delta
	beta(2) = (y(3)-y(1))/delta
	beta(3) = (y(1)-y(2))/delta
	gamma(1) = (x(3)-x(2))/delta
	gamma(2) = (x(1)-x(3))/delta
	gamma(3) = (x(2)-x(1))/delta
	
!     --- Fill the element matrix as defined in the Lecture Notes

	do j = 1, 3
	  do i = 1, 3
	    elem_mat(i,j) = 0.5d0*k*abs(delta)*(beta(i)*beta(j)+gamma(i)*gamma(j))
	  end do
	end do
	
!     --- The element vector is zero
  
	elem_vec(1:3) = 0d0
	
	
      else if ( itype==3 ) then   		!--- Boundary element for Omega_A
	
!--- Fill the element matrix
	suv = 0d0
	
	do i = 1, 3
	  do j = 1,3
	    suv(i,j) = alpha * abs(x(2)-x(1) + y(1)-y(2))/6
	  end do
	end do
	
	elem_mat = suv
	
	do i = 1,3
	    elem_mat(i,i) = suv(i,i) * 2d0
	end do
	
!--- Fill the element vector

	elem_vec(1:2) = alpha * T_inf * 0.5d0 * abs(x(2)-x(1) + y(1)-y(2))
	
      else if ( itype==4 ) then   		!--- Boundary element for Omega_B
	
!--- Fill the element matrix
	suv = 0d0
	
	do i = 1, 3
	  do j = 1,3
	    suv(i,j) = alpha * abs(x(2)-x(1) + y(1)-y(2))/6
	  end do
	end do
	
	elem_mat = suv
	
	do i = 1,3
	    elem_mat(i,i) = suv(i,i) * 2d0
	end do
	
!--- Fill the element vector

	elem_vec(1:2) = alpha * T_inf * 0.5d0 * abs(x(2)-x(1) + y(1)-y(2))
	
      end if
      
end subroutine elemsubr
\end{lstlisting}

\subsection{Problem definition}\label{ap:prb}
\begin{lstlisting}
#*****************************************************************************
#
# File: lab4-2.prb
#
# Usage: sepcompexe < lab4-2.prb
#
# It has been supposed that the following actions have been carried out
# with success:
#
# sepmesh lab4-2.msh
# compile elemsubr.f90
# seplink sepcompexe
#*****************************************************************************
#
#
#
# Problem definition
#
problem
  types					# Define type numbers per element group
    elgrp1 = (type=1)			# Element group 1: itype = 1
    elgrp2 = (type=2)			# Element group 2: itype = 2
  natbouncond				# Define types of natural boundary
					# conditions
    bngrp1 = (type=3)			# Boundary element group 1: itype = 3
    bngrp2 = (type=4)			# Boundary element group 2: itype = 4
  bounelements				# Define where natural boundary
					# conditions are present
  belm1 = curves(c2)			# Boundary element group 1 is
					# defined on c3
  essboundcond				# Define where essential boundary
					# conditions are defined (not there
					# value)
	curves(c1)			# Left boundary	
end
#
# Define the structure of the main program
#
structure
  # Define the structure of the large matrix
  matrix_structure: symmetric 		# a symmetric profile matrix is used
  
# Define non-zero essential boundary conditions
  prescribe_boundary_conditions potential = 1, curves(c1) # T_0 = 1, essential B.C.
  solve_linear_system, potential
  print potential
  plot_contour potential
plot_coloured_levels potential
end
end_of_sepran_input

\end{lstlisting}